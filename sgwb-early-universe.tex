\documentclass[10pt]{article}
\usepackage{amsmath}
\usepackage[colorlinks=true]{hyperref}
\usepackage{cancel}

\begin{document}
\title{Solutions to SGWB Early Universe problem set}
\author{Md Arif Shaikh}
\date{\today}
\maketitle

\begin{enumerate}
    \item \underline{Derivation of linearized Ricci tensor around a Minkowski background}\\\\
    The space-time metric $g_{\mu\nu}$ could be written in terms of small perturbation $h_{\mu\nu}$ around the Minkowski spacetime $\eta_{\mu\nu}$ in the following way,
    \begin{equation}
        \label{eq:metric-perturbation}
        g_{\alpha\beta} = \eta_{\alpha\beta} + h_{\alpha\beta},\qquad |h_{\alpha\beta}|\ll 1
    \end{equation}
    using this we can expand the Christoffel symbols. Christoffel symbols in terms of $g_{\mu\nu}$ is given by
    \begin{equation}
        \label{eq:Christofell-symbols}
        \Gamma^{\gamma}_{\alpha\beta} = \frac{1}{2}g^{\gamma\delta}(\partial_\alpha g_{\beta\delta} + \partial_\beta g_{\alpha\delta} - \partial_\delta g_{\alpha\beta}).
    \end{equation}
    From eq. (\ref{eq:metric-perturbation}), one can get $g^{\alpha\beta} = \eta^{\alpha\beta} - h^{\alpha\beta}$. Since derivative of $\eta_{\mu\nu}$ is zero, only terms inside the parenthesis that would contribute are the derivatives of $h_{\alpha\beta}$. Also, since we want terms up to $\mathcal{O}(h)$, only term that would contribute from $g^{\alpha\beta}$ would be $\eta^{\alpha\beta}$. Therefore the linearized $\Gamma$ could be written as
    \begin{equation}
        \label{eq:linearized-Gamma}
        \Gamma^{\gamma}_{\alpha\beta} = \frac{1}{2}\eta^{\gamma\delta}(\partial_\alpha h_{\beta\delta} + \partial_\beta h_{\alpha\delta} - \partial_\delta h_{\alpha\beta})
    \end{equation}
    Now we proceed to compute the linearized Riemann tensor. The general expression of Riemann tensor in terms of the Christoffel symbols is given by
    \begin{equation}
        \label{eq:Riemman-tensor}
        R_{\alpha\beta\gamma}^{~~~~\delta} = -\partial_\alpha\Gamma^{\delta}_{\beta\gamma} + \partial_\beta\Gamma^{\delta}_{\alpha\gamma} - \Gamma^{\delta}_{\alpha\mu}\Gamma^{\mu}_{\beta\gamma} + \Gamma^{\delta}_{\beta\mu}\Gamma^{\mu}_{\alpha\gamma}.
    \end{equation}
    From eq. (\ref{eq:linearized-Gamma}), we see that the $\Gamma$ are of $\mathcal{O}(h)$ and therefore the $\Gamma\Gamma$ terms are of $\mathcal{O}(h^2)$ and hence can be neglected. So we are left with the first derivative of $\Gamma$ terms. We get $\partial_\beta \Gamma^{\delta}_{\alpha\gamma}$ and then exchange $\alpha$ and $\beta$ to get $\partial_\alpha\Gamma^{\delta}_{\beta\gamma}$.
    
    \begin{align}
        \partial_\beta \Gamma^{\delta}_{\alpha\gamma} = \frac{1}{2}\eta^{\delta\lambda} (\cancel{\partial_{\alpha} \partial_{\beta} h_{\lambda\gamma}} + \partial_\beta\partial_\gamma h_{\alpha\lambda} - \partial_\beta \partial_\lambda h_{\alpha\gamma}) \\
        - \partial_\alpha \Gamma^{\delta}_{\beta\gamma} = \frac{1}{2}\eta^{\delta\lambda} (- \cancel{\partial_{\beta} \partial_{\alpha} h_{\lambda\gamma}} - \partial_\alpha\partial_\gamma h_{\beta\lambda} + \partial_\alpha \partial_\lambda h_{\beta\gamma})\\
        \hline \nonumber \\
        R_{\alpha\beta\gamma}^{~~~~\delta} = \partial_\beta \Gamma^{\delta}_{\alpha\gamma} - \partial_\alpha \Gamma^{\delta}_{\beta\gamma} = \frac{1}{2}\eta^{\delta\lambda}(\partial_\beta \partial_\gamma h_{\alpha\lambda} - \partial_\beta\partial_\lambda h_{\alpha\gamma} - \partial_\alpha\partial_\gamma h_{\beta\lambda} + \partial_\alpha\partial_\lambda h_{\beta\gamma})
    \end{align}
    This provides the linearized Riemann tensor as
    \begin{equation}
        \label{eq:linearized-Riemann-tensor}
        \boxed{R_{\alpha\beta\gamma\delta} = \frac{1}{2}(\partial_\beta\partial_\gamma h_{\alpha\delta} - \partial_\beta \partial_\delta h_{\alpha\gamma} - \partial_\alpha\partial_\gamma h_{\beta\delta} + \partial_\alpha\partial_\delta h_{\beta\gamma})}\qquad \textrm{({\bfseries Proved})}
    \end{equation}
    
    \item \underline{Derivation of linearized equation of motion (Linearized Einstein equation)}\\\\
    We have the linearized Riemann tensor in eq. (\ref{eq:linearized-Riemann-tensor}). All we need now is to compute the linearized Ricci tensor and the Ricci scalar and get the linearized Einstein tensor using that.
    
    To get the Ricci tensor we contract the first index with third index, i. e., $g^{\alpha\gamma}R_{\alpha\beta\gamma\delta} = \eta^{\alpha\gamma}R_{\alpha\beta\gamma\delta} = R_{\beta\delta}$ since Riemann tensor is already of $\mathcal{O}(h)$, to get
    \begin{equation}
        \label{eq:Ricci-tensor}
        R_{\alpha\beta} = \frac{1}{2}(-\partial_\mu \partial^\mu h_{\alpha\beta} + \partial_\alpha\partial^\mu h_{\mu\beta} + \partial_\beta\partial^\mu h_{\mu\alpha} - \partial_\alpha \partial_\beta h)
    \end{equation}
    where $h$ is the trace of $h_{\alpha\beta}$, i. e., $h = h^\alpha_\alpha$.
    
    Next we get the Ricci scalar by contracting the indices of $R_{\alpha\beta}$, i. e., $R = g^{\alpha\beta}R_{\alpha\beta} = \eta^{\alpha\beta} R_{\alpha\beta}$.
    \begin{equation}
        \label{eq:Ricci-scalar}
        R = \frac{1}{2}(-\partial_\mu\partial^\mu h + \partial_\alpha \partial^\mu h_{\mu}^{\alpha} + \partial_\beta \partial^{\mu} h_{\mu}^{\beta} - \partial_\alpha\partial^\alpha h) = -\eta^{\alpha\beta}\partial_{\alpha}\partial_\beta h + \partial_\alpha\partial_\beta h^{\alpha\beta}.
    \end{equation}
    where we have changed the dummy indices as per our choice.
    
    Now that we have the Ricci tensor and Ricci scalar we need to insert these in the Einstein tensor
    \begin{equation}
        \label{eq:Einstein-tensor}
        G_{\alpha\beta} = R_{\alpha\beta} - \frac{1}{2}g_{\alpha\beta}R = R_{\alpha\beta} - \frac{1}{2}\eta_{\alpha\beta}R.
    \end{equation}
    We get the following after substituting eq. (\ref{eq:Ricci-tensor}) and eq. (\ref{eq:Ricci-scalar}) in eq. (\ref{eq:Einstein-tensor}) above,
    
    \begin{equation}
        \label{eq:linearized-Einstein-tensor-in-h}
        G_{\alpha\beta} = \frac{1}{2}(-\eta^{\mu\nu}\partial_\mu\partial_\nu h_{\alpha\beta} + \partial_\alpha\partial^\mu h_{\mu\beta} + \partial_\beta\partial^\mu h_{\mu\alpha} - \partial_\alpha \partial_\beta h) - \frac{1}{2}\eta_{\alpha\beta}(-\eta^{\mu\nu}\partial_\mu\partial_\nu h + \partial_\mu\partial_\nu h^{\mu\nu})
    \end{equation}
    
\end{enumerate}
\end{document}